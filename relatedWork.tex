\section{Related Work}
Most meeting reminder applications have a set notification time for the user, and do not take transit into consideration. During the research conducted prior to designing \input{appName.tex}, it was evident that there was little work done on dynamically reminding a user to leave on time and updating the ETA as they are heading to the meeting. This is a feature that users would use as many people have been late to meetings, and if they missed their Outlook notification they don't know what time they will be arriving at the meeting to let anyone know. \input{appName.tex} solves this by learning how long it takes the user to go to the meeting and giving them a notification of when to leave so that they arrive on time. If they are late, and are sprinting to the meeting, it will let them know what time they will arrive at that pace. This is not seen in other mobile solutions and it is an area which work is needed in.
\subsection{Google Assistant}
Google Assistant will display a leave by time when an event is loaded into Google Calendar. This notification takes traffic into consideration, but assumes the user is driving by car. Many WPI students have meetings on campus, and will walk (or run/bike) to their meetings rather than drive there. Therefore, the use case for this notification is when attending a meeting or appointment in which a car will be used to get there, and not for walking to a project meeting on a small college campus from a dorm or nearby apartment.\footnote{https://assistant.google.com/intl/en\_us} \footnote{https://play.google.com/store/apps/details?id=com.google.\\android.googlequicksearchbox}

\subsection{Outlook} Outlook is a widely used calendar system, which supports notifications for meetings. Most WPI students will use Outlook Calendar, as it is connected to their WPI email addresses. Many meetings will be scheduled in Outlook Calendar and it does include built-in notification options. The notification options include a set amount of time before the meeting, such as 15 minutes prior to the meeting. This is handy if you know how much time it will take you to go to the meeting, but if wrong, you may end up too early or late. 
