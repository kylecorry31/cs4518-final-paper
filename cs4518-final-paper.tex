% This is "sig-alternate.tex" V1.8 June 2007 Modified for SOUPS 2011
% This file should be compiled with V2.3 of "sig-alternate.cls" June 2007
%
% This example file demonstrates the use of the 'sig-alternate.cls'
% V2.3 LaTeX2e document class file. It is for those submitting
% articles to ACM Conference Proceedings WHO DO NOT WISH TO
% STRICTLY ADHERE TO THE SIGS (PUBS-BOARD-ENDORSED) STYLE.
% The 'sig-alternate.cls' file will produce a similar-looking,
% albeit, 'tighter' paper resulting in, invariably, fewer pages.
%
% ----------------------------------------------------------------------------------------------------------------
% This .tex file (and associated .cls V2.3) produces:
%       1) The Permission Statement
%       2) The Conference (location) Info information
%       3) The Copyright Line with ACM data
%       4) NO page numbers
%
% as against the acm_proc_article-sp.cls file which
% DOES NOT produce 1) thru' 3) above.
%
% Using 'sig-alternate.cls' you have control, however, from within
% the source .tex file, over both the CopyrightYear
% (defaulted to 200X) and the ACM Copyright Data
% (defaulted to X-XXXXX-XX-X/XX/XX).
% e.g.
% \CopyrightYear{2007} will cause 2007 to appear in the copyright line.
% \crdata{0-12345-67-8/90/12} will cause 0-12345-67-8/90/12 to appear in the copyright line.
%
% ---------------------------------------------------------------------------------------------------------------
% This .tex source is an example which *does* use
% the .bib file (from which the .bbl file % is produced).
% REMEMBER HOWEVER: After having produced the .bbl file,
% and prior to final submission, you *NEED* to 'insert'
% your .bbl file into your source .tex file so as to provide
% ONE 'self-contained' source file.
%
% ================= IF YOU HAVE QUESTIONS =======================
% Questions regarding the SIGS styles, SIGS policies and
% procedures, Conferences etc. should be sent to
% Adrienne Griscti (griscti@acm.org)
%
% Technical questions _only_ to
% Gerald Murray (murray@acm.org)
% ===============================================================
%
% For tracking purposes - this is V1.8 - June 2007

% --- Start page size ---
%Please use the following format  
\documentclass[twoside,letterpaper]{soups} 
\pdfpagewidth=8.5truein 
\pdfpageheight=11truein 
% --- End page size ---


\usepackage{graphicx}
\renewcommand{\topfraction}{0.99} % be more aggressive about text around floats
\renewcommand{\floatpagefraction}{0.99}
\pagestyle{plain} % page numbers

\begin{document}
%
% --- Author Metadata here ---
\conferenceinfo{Symposium on Usable Privacy and Security
  (SOUPS)}{2013, July 24--26, 2013, Newcastle, UK.}
\CopyrightYear{2013} % Allows default copyright year (200X) to be over-ridden - IF NEED BE.
%\crdata{0-12345-67-8/90/01}  % Allows default copyright data (0-89791-88-6/97/05) to be over-ridden - IF NEED BE.
% --- End of Author Metadata ---

\title{Remind Me}
%
% You need the command \numberofauthors to handle the 'placement
% and alignment' of the authors beneath the title.
%
% For aesthetic reasons, we recommend 'three authors at a time'
% i.e. three 'name/affiliation blocks' be placed beneath the title.
%
% NOTE: You are NOT restricted in how many 'rows' of
% "name/affiliations" may appear. We just ask that you restrict
% the number of 'columns' to three.
%
% Because of the available 'opening page real-estate'
% we ask you to refrain from putting more than six authors
% (two rows with three columns) beneath the article title.
% More than six makes the first-page appear very cluttered indeed.
%
% Use the \alignauthor commands to handle the names
% and affiliations for an 'aesthetic maximum' of six authors.
% Add names, affiliations, addresses for
% the seventh etc. author(s) as the argument for the
% \additionalauthors command.
% These 'additional authors' will be output/set for you
% without further effort on your part as the last section in
% the body of your article BEFORE References or any Appendices.

\numberofauthors{5} %  in this sample file, there are a *total*
% of EIGHT authors. SIX appear on the 'first-page' (for formatting
% reasons) and the remaining two appear in the \additionalauthors section.
%
\author{
% You can go ahead and credit any number of authors here,
% e.g. one 'row of three' or two rows (consisting of one row of three
% and a second row of one, two or three).
%
% The command \alignauthor (no curly braces needed) should
% precede each author name, affiliation/snail-mail address and
% e-mail address. Additionally, tag each line of
% affiliation/address with \affaddr, and tag the
% e-mail address with \email.
%
% 1st. author
\alignauthor John Stegeman\\
       \affaddr{Worcester Polytechnic Institute}\\
       \email{jtstegeman@wpi.edu}
% 2nd. author
\alignauthor Kyle Corry\\
       \affaddr{Worcester Polytechnic Institute}\\
       \email{kncorry@wpi.edu}
% 3rd. author
\alignauthor Zoraver Kang\\
        \affaddr{Worcester Polytechnic Institute}\\
       \email{zkang@wpi.edu}
\and  % use '\and' if you need 'another row' of author names
% 4th. author
\alignauthor Manuel (Jack) Gonsalves\\
        \affaddr{Worcester Polytechnic Institute}\\
       \email{msgonsalves@wpi.edu}
% 5th. author
\alignauthor Alex Taglieri\\
        \affaddr{Worcester Polytechnic Institute}\\
       \email{ataglieri@wpi.edu}
}


\maketitle
\begin{abstract}
Many students find themselves distracted by studying, homework, games, etc. and then realize that they should have already left to attend a meeting - they will now be late, again. Another reason for being late is that users underestimate the time it will take them to walk to a location because it looks close on a map to them ~\cite{raghubir1996crow}. Being late for meetings is embarrassing, especially if it happens often, and is counterproductive for both the student who is late and those waiting for them. Time is very important at college, so being on time to a meeting could save a few minutes in the end as the meeting can start on time. Remind Me is designed to notify the user of when they have to leave to arrive on time based on the speed that the app learns they normally walk at. It also recomputes the ETA on the fly as the person is making their way to the meeting so they know they will be on time or not based on their current mode of transit and speed.
\end{abstract}

% A category with the (minimum) three required fields
\category{H.4}{Information Systems Applications}{Miscellaneous}
%A category including the fourth, optional field follows...
\category{D.2.8}{Software Engineering}{Metrics}[complexity measures, performance measures]

% Your general terms must be any of the following 16 designated terms:
% Algorithms, Management, Measurement, Documentation, Performance,
% Design, Economics, Reliability, Experimentation, Security, Human
% Factors, Standardization, Languages, Theory, Legal Aspects, Verification
\terms{Human Factors}

% You are on your own with keywords
\keywords{Meetings, late, eta, gps, location, reminder}

\section{Introduction}

\section{Related Work}
Most meeting reminder applications have a set notification time for the user, and do not take transit into consideration. During the research conducted prior to designing Remind Me, it was evident that there was little work done on dynamically reminding a user to leave on time and updating the ETA as they are heading to the meeting. This is a feature that users would use as many people have been late to meetings, and if they missed their Outlook notification they don't know what time they will be arriving at the meeting to let anyone know. Remind Me solves this by learning how long it takes the user to go to the meeting and giving them a notification of when to leave so that they arrive on time. If they are late, and are sprinting to the meeting, it will let them know what time they will arrive at that pace. This is not seen in other mobile solutions and it is an area which work is needed in.
\subsection{Google Assistant}
Google Assistant will display a "leave by" time when an event is loaded into Google Calendar. This notification takes traffic into consideration, but assumes the user is driving by car. Many WPI students have meetings on campus, and will walk (or run/bike) to their meetings rather than drive there. Therefore, the use case for this notification is when attending a meeting or appointment in which a car will be used to get there, and not for walking to a project meeting on a small college campus from a dorm or nearby apartment.\footnote{https://assistant.google.com/intl/en\_us} \footnote{https://play.google.com/store/apps/details?id=com.google.\\android.googlequicksearchbox}

\subsection{Outlook} Outlook is a widely used calendar system, which supports notifications for meetings. Most WPI students will use Outlook Calendar, as it is connected to their WPI email addresses. Many meetings will be scheduled in Outlook Calendar and it does include built-in notification options. The notification options include a set amount of time before the meeting, such as 15 minutes prior to the meeting. This is handy if you know how much time it will take you to go to the meeting, but if wrong, you may end up too early or late. 


\section{Approach/Methodology}

\subsection{ETA System}
To calculate the time to a given location, an ETA system was designed which used an approximate distance calculation to estimate the distance the user would have to travel, and the speed of the user to determine how long it would take them to arrive at the location. The distance calculation involved an 'as the crow flies' representation of the distance ~\cite{androidlocation} with a multiple of $\frac{\pi}{2}$ (see Equation~\ref{eq:distanceMultiplier}, which will give an approximate distance based on the sinusoidal properties of pathways to the destination ~\cite{numberphile2014pi} ~\cite{edwards2004crow} ~\cite{lembo2003crow}. This was done to conserve battery and privacy (as no network requests are needed) while giving a fast approximation of the distance to travel. The system will also learn from the user's current speed an activity to get a better estimation of how fast they typically walk, run, bike, or drive to their meetings. This is done using an exponential moving average with an alpha of 0.5 and storing the result in the SharedPreferences (see Equation~\ref{eq:ema})\footnote{The alpha of 0.5 was chosen based on the way operating systems predict burst time, and a reason for choosing an EMA was based on an article by Williams ~\cite{williams1998urban}} ~\cite{williams1998urban}. Using that speed and distance estimate, the time can be calculated by dividing the distance by the speed (see Equation~\ref{eq:time}). This will give an approximate travel time in seconds, which is then interpreted by higher level components of the application.

\begin{equation}\label{eq:distanceMultiplier}
	dist(x, y) = \frac{\pi}{2} straightLineDistance(x, y)
\end{equation}

\begin{equation}\label{eq:ema}
	\tau_{n+1} = \alpha t_n + (1 - \alpha) \tau_n
\end{equation}

\begin{equation}\label{eq:time}
	t = \frac{x}{v}
\end{equation}

\section{Implementation}
\subsection{Points}
The following Android features are implemented in the app with their corresponding feature score based on the initial grading rubric.  They are discussed in further detail below.

\begin {itemize}
	\item >5 UI Screens (4 pts)
	\item Location Sensing (4 pts)
	\item Sending SMS (4 pts)
	\item Maps (4 pts)
	\item Custom Geofencing (6 pts)
	\item Activity Recognition (6 pts)
	\item Notifications
	\item Android System Alarms
\end {itemize}

\subsection{UI}
The UI components were all developed in Android studio with the standard drag and drop interface. 

\subsection{Maps}
The maps shown in the app use the usual Google Play implementation that we used previously in Lab 3. Nothing to special here, the location of the meeting is taken from the database and passed in to show the location on the map.

\subsection{Database}
At the core of the app is the SQLite database.  Our database holds all the alarms that the use has scheduled. The class for an alarm is AlarmObject and it holds the scheduled meeting time, the meeting location in latitude and longitude, the meeting name, the current warnings of when to leave issued for the alarm (to avoid spamming the user), and the list of phone numbers to send a late message to if necessary. We utilize custom cursors to give us seamless integration when querying the database. 

\subsection{Notifications and SMS}
Our notification system is implemented with a singleton pattern that registers itself when the operating system when it is instantiated. This pattern hides most of the low level details about how the notifications are setup and prevents multiple parts of the app from creating multiple copies of the notification system. The singleton exposes a shutdown method that deregisters the notification system from the OS. The singleton also exposes a single method “sendNotification(Context ctx, String title, String content)” that sends the user a notification with the specified title and content. Similarly the SMS portion of it exposes a method “sendSMS(String number, String message)” that simply sends an sms message to whatever the phone number is with the message contents of the parameter message.

\subsection{Location and Geofencing}
We track location with a similar singleton pattern as mentioned above for notifications. The singleton keeps track of the current location and exposes it via a method. To enable faster starting of location tracking when the app is restarted, the last known location is saved as a preference for the app. When the app receives the kill hook from the OS the current location is written into the preferences. When the app starts again, it checks to see if there was a last known location in preferences, and if there was uses that one as the initial location until the location service received an updated location. Custom Geofencing is implemented in the same class. Other components of the app can pass geofences in and when the user goes within the specified distance of the target, the location singleton sends a broadcast.

\subsection{Activity Recognition}
We use the Google Play API to get activity information. The activity information is also cached in the app preferences to enable faster start times to get a decent activity guess when the app is started. The activity recognition is done as a simple broadcast receiver since we don't really need to call any methods other than retrieving the static variable representing the last known activity.

\subsection{ETA Calculation}
Our ETA calculations are implemented with the strategy design pattern.  That pattern lets the user select the best estimator for them, whether it is a simple distance based estimator or one that uses machine learning to extract the average trip time and make a best guess for the user. Regardless of the implementation chosen, they all make an estimation of how long it will take to get from Location A to Location B at an activity level. Taking into account the current activity level helps provide accurate notifications on if the user will be late. Since people move faster when they run, bike or drive, those activities lower the estimated ETA; if a user is running late and starts running at a pace that will let them reach the meeting on time, the app won't send a notification to them or the others in the meeting that they are late. 

The simplest time estimator, and often times the best, is a simple as-the-crow-flies estimator. It finds the linear distance between locations, then multiplies that with an experimentally derived constant to calculate the approximate upper bound for the distance.  Once that distance is known, the estimator divides that distance with rates determined by the current activity level to guess at the estimated arrival time.

\subsection{Android Alarm System Scheduling}
We use a relatively simple scheduling algorithm. Each alarm registers itself with the Android OS to receive a broadcast when that meeting is within the next 60 minutes. At that point the scheduling algorithm uses the ETA Calculation, Location, and Current Activity components to get an estimate of the estimated travel time from the current location. The scheduler will then schedule its next broadcast from the OS at the midpoint between the 5 minute warning point to leave for the meeting and the current time. This strategy allows the app to reliably cope with the user being in many locations and distances away before the meeting starts.  Once the estimated departure is within 7 minutes, the alarm schedules itself at 1 minute intervals to provide updates to the user. At 5 minutes and 1 minute before the calculated leave time, the app sends notifications to the user. If the user ignores these and misses the calculated leave time, the app prompts them to be on their way. If the user still doesn't leave within two minutes of that point, the app sends an SMS notification to the others at the meeting that the person will be late. If the user runs or bikes to make up time, the app lets them know they are back on track to make the meeting. This scheduling/notification structure works reasonably well without overwhelming the user with notifications.

\section{Evaluation/Results}
Our app was successful in alerting its users to their meetings. One of our team members would have shown up late to our presentation had they not been using Remind Me (done for testing purposes of course). The app's unique ETA estimating algorithm has built-in padding, so it always overestimates the time required for the user to get to the destination. 

Members of the team used the app for navigating to various meetings, including a trip from Founders to Lee Street. None of us have bicycles so we were unable to test the biking feature, but that part of the app mostly uses Google's Activity Recognition and Google Maps anyway. Our app estimates travel time for bicycles as being roughly half that of walking.

Walking directions were thoroughly evaluated. The app always left enough time for the user to arrive early. 


\section{Discussion}

\section{Conclusion}
In this paper, we presented the Remind Me Android application which would help to prevent students from being late to meetings by notifying them of when to leave and what time they will arrive. We discussed methods of how we calculated how long it will take the user to make it to the meeting and how we notify them of when to leave and when they will be late. To our knowledge, this is the first system of its kind which knows how the user is going to the meeting and learns how fast they normally travel to give them insights into what time they need to leave to be there on time. Of course, we only had around 2-3 limited weeks to work on the app in the so there is always room for improvement.

\section{Future Work}
Future versions of this app could incorporate many additional features to improve the eta estimates for the user. Future versions could add calendar integration so that the app can pull meeting times and locations from the user's outlook calendar. With this integration, an additional useful feature would be automatically converting a location to usable GPS coordinates for the app to use. A final major future item to do to the app would be to add more advanced machine learning algorithms to automatically adjust estimated travel times for particular users. If a user walks really fast or slow, the app would recognize this and adjust the estimated travel times accordingly to give better alerts on when to leave.

Other improvements may be GPS smoothing to avoid jumping around with bad GPS signals. Finally, users may find it useful to be able to set their own default speed (although the system will learn it) and be able to identify which forms of transportation that they may use. 


\bibliographystyle{abbrv}
\bibliography{sigproc}

\end{document}
