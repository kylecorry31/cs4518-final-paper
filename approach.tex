\section{Approach/Methodology}

\subsection{Determining the Problem}
In order to determine that this is a problem, we asked several people on campus, and from our own experience and found that many people arrive a few minutes late to meetings, and cause aggravation among group members. This seems to be an issue in classes with large groups as well, such as software engineering and even in club meetings. Therefore, it was not hard to figure out that this was actually a problem at WPI. 

\subsection{Research Conducted}
We examined systems which are currently in place, such as Outlook Calendar and Google Now. Using these systems, we found that it would be best to have a combination of the two. Google Now has features in which it tells the user what time they will need to leave to make a meeting on time, but it relied on the fact that the user would be driving to a meeting, we also found that this feature was limited to users who have installed Google Now or have a Google phone. Outlook Calendar, which most WPI students use, sends out reminders a certain number of minutes before a meeting, for example 15 minutes. This is better fitted for people walking to meetings, like most WPI students do, but if you leave when you get that notification you may be too early or too late. Plus if you wait a little before leaving, you don't know what time you are going to arrive at the meeting. Using these two systems, we determined that it would be best to combine them and make a single system which alerts you and lets you know what time you will be back.

\subsection{ETA System}
To calculate the time to a given location, an ETA system was designed which used an approximate distance calculation to estimate the distance the user would have to travel, and the speed of the user to determine how long it would take them to arrive at the location. The distance calculation involved an 'as the crow flies' representation of the distance ~\cite{androidlocation} with a multiple of $\frac{\pi}{2}$ (see Equation~\ref{eq:distanceMultiplier}, which will give an approximate distance based on the sinusoidal properties of pathways to the destination ~\cite{numberphile2014pi} ~\cite{edwards2004crow} ~\cite{lembo2003crow}. This was done to conserve battery and privacy (as no network requests are needed) while giving a fast approximation of the distance to travel. The system will also learn from the user's current speed an activity to get a better estimation of how fast they typically walk, run, bike, or drive to their meetings. This is done using an exponential moving average with an alpha of 0.6 and storing the result in the SharedPreferences (see Equation~\ref{eq:ema})\footnote{The alpha of 0.6 was chosen based on an article by Williams ~\cite{williams1998urban}} ~\cite{williams1998urban}. Using that speed and distance estimate, the time can be calculated by dividing the distance by the speed (see Equation~\ref{eq:time}). This will give an approximate travel time in seconds, which is then interpreted by higher level components of the application.

\begin{equation}\label{eq:distanceMultiplier}
	dist(x, y) = \frac{\pi}{2} straightLineDistance(x, y)
\end{equation}

\begin{equation}\label{eq:ema}
	\tau_{n+1} = \alpha t_n + (1 - \alpha) \tau_n
\end{equation}

\begin{equation}\label{eq:time}
	t = \frac{x}{v}
\end{equation}